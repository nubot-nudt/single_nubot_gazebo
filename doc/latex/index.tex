\subparagraph*{\#\#\#\#\#\#\#\#\#\#\# Description}

This simulation system can simulate one robot soccer player for Robo\-Cup Middle Size League. It can be adapted for other purposes.

\subparagraph*{\#\#\#\#\#\#\#\#\#\#\# Recommended Operating Environment}


\begin{DoxyEnumerate}
\item Ubuntu 14.\-04;
\item R\-O\-S Indigo;
\item Gazebo 5.\-0 or 5.\-1
\item gazebo\-\_\-ros\-\_\-pkgs N\-O\-T\-E\-: If you choose \char`\"{}desktop-\/full\char`\"{} install of R\-O\-S Indigo, there is a Gazebo 2.\-0 included initially. In order to install Gazebo 5.\-0/5.1, you should first remove Gazebo 2.\-0 by running\-: \$ sudo apt-\/get remove gazebo2$\ast$ Then you should be able to install Gazebo 5.\-0 now. To install gazebo\-\_\-ros\-\_\-pkgs compatible with Gazebo 5.\-0/5.1, run this command\-: \$ sudo apt-\/get install ros-\/indigo-\/gazebo5-\/ros-\/pkgs ros-\/indigo-\/gazebo5-\/ros-\/control
\end{DoxyEnumerate}

Other versions of Ubuntu, R\-O\-S or Gazebo may also work, but we have not tested yet.

\subparagraph*{\#\#\#\#\#\#\#\#\#\#\# Complie}


\begin{DoxyEnumerate}
\item Go to the package root directory (single\-\_\-nubot\-\_\-gazebo\-\_\-ws)
\item If you already have C\-Make\-Lists.\-txt in the \char`\"{}src\char`\"{} folder, then you can skip this step. If not, run these commands\-: \$ cd src \$ catkin\-\_\-init\-\_\-workspace \$ cd ..
\item \$ ./configure
\item \$ catkin\-\_\-make --pkg nubot\-\_\-common
\item \$ catkin\-\_\-make
\end{DoxyEnumerate}

\subparagraph*{\#\#\#\#\#\#\#\#\#\#\# Tutorial}

\subsubsection*{Part I. Overview}

The robot movement is realized by a Gazebo model plugin which is called \char`\"{}\-Nubot\-Gazebo\char`\"{} generated by source files \char`\"{}nubot\-\_\-gazebo.\-cc\char`\"{} and \char`\"{}nubot\-\_\-gazebo.\-hh\char`\"{}. Basically the essential part of the plugin is realizing basic motions\-: omnidirectional locomotion, ball-\/dribbling and ball-\/kicking.

Basically, this plugin subscribes to topic \char`\"{}nubotcontrol/velcmd\char`\"{} for omnidirecitonal movement and subscribes to service \char`\"{}\-Ball\-Handle\char`\"{} and \char`\"{}\-Shoot\char`\"{} for ball-\/dribbling and ball-\/kicking respectively. You can customize this code for your robot based on these messages and services as a convenient interface.

As for ball-\/dribbling, there are three ways for a robot to dribble a ball, i.\-e. (a). Setting ball pose continually. This is the most accurate one; nubot would hardly lose control of the ball, but the visual effect is not very good(the ball does not rotate). (b). Setting ball secant velocity. This is less acurate than method (a) but better than method (c). (c). Setting ball tangential velocity. This is the least accurate. If the robot moves fast, such as 3 m/s, it would probably lose control of the ball. However, this method achieves the best visual effect under low-\/speed conditon. By default, we use method (c) for ball-\/dribbling.

As for Gaussian noise, By default, Gaussian noise is N\-O\-T added, but you can add it by changing the flag in \hyperlink{nubot__gazebo_8cc}{nubot\-\_\-gazebo.\-cc} in function update\-\_\-model\-\_\-info();

Please read the paper \char`\"{}\-Weijia Yao et al., A Simulation System Based on R\-O\-S and Gazebo for Robo\-Cup Middle Size League, 2015\char`\"{} for more information.

\subsubsection*{Part I\-I. Single robot automatic movement}

The robot will do motions according to states transfer graph. Steps are as follows\-:
\begin{DoxyEnumerate}
\item Go to the package root directory (single\-\_\-nubot\-\_\-gazebo\-\_\-ws)
\item source the setup.\-bash file\-: \$ source devel/setup.\-bash Note\-: Every time you open a new terminal, you have to do this step. You can also write this command into the $\sim$/.bashrc file so that you don't have to source it every time.
\item \$ roslaunch nubot\-\_\-gazebo sdf\-\_\-nubot.\-launch The robot rotates and translates with trajector planning. That is, the robot accelerates at constant acceleration and stays at constant speed when it reaches the maximum velocity.
\end{DoxyEnumerate}

\subsubsection*{Part I\-I\-I. Keyboad control robot movement}


\begin{DoxyEnumerate}
\item In \hyperlink{nubot__gazebo_8cc}{nubot\-\_\-gazebo.\-cc}, comment \char`\"{}nubot\-\_\-auto\-\_\-control();\char`\"{} and uncomment \char`\"{}nubot\-\_\-be\-\_\-control();\char`\"{} in function Update\-Child().
\item Compile again and follow steps 1-\/3 listed in Part I\-I.
\item \$ rosrun nubot\-\_\-gazebo nubot\-\_\-teleop\-\_\-keyboard
\end{DoxyEnumerate}

\subsubsection*{Part I\-V. Appendix}

a. to launch an empty soccer field\-: \$ roslaunch nubot\-\_\-gazebo empty\-\_\-field.\-launch b. to launch the simulation world with rqt\-\_\-plot of nubot or ball's velocity\-: \$ roslaunch nubot\-\_\-gazebo sdf\-\_\-nubot.\-launch plot\-:=true 